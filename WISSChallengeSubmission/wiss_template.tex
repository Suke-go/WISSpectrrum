\documentclass[twoside]{wiss}

\usepackage{ascmac}
\usepackage[dvipdfmx]{graphicx}
\usepackage{nidanfloat}
\usepackage{multicol}
\usepackage{color}
\usepackage{flushend}
\usepackage{url}

\journalhead{WISSpectrum: 論文探索支援プラットフォーム}

\begin{document}

\title{WISSpectrum: HCI研究者のための論文探索支援プラットフォーム}
\etitle{WISSpectrum: A Discovery Platform for HCI Researchers}
\author{清水 紘輔\affil{筑波大学}}

\begin{abstract}
国内のHCI研究コミュニティにおいては,大規模言語モデル(LLM)を活用した論文サーベイ支援が急速に普及しているが,検索結果の羅列や要約提示にとどまり,研究者が課題設定や評価設計を主体的に検討するための導線が不足している。本稿では,インタラクティブシステムとソフトウェアに関するワークショップ(WISS)の論文群を対象に,背景・目的・手法・評価の各セクションを構造化して提示し,意味的距離とACM CCS概念に基づく多層的探索を可能にするプラットフォーム「WISSpectrum」を提案する。PDF収集から要約生成,埋め込み計算,概念分類,可視化までを統合した前処理パイプラインを整備し,研究着想のブラッシュアップ,査読準備,教育支援を想定したプロトタイプを実装した。HCI研究の議論を再活性化するための設計指針と今後の課題を報告する。
\end{abstract}

\maketitle

\section{はじめに}
HCI分野では広範なテーマと評価手法が交錯するため,既存研究の位置づけをつかむには単なる検索結果以上の構造化が求められる。とりわけWISSはアイデア指向が強く,過去の議論を参照しながら未来志向の提案を磨く文化を持つ。近年はLLMを用いた要約やQAが普及したものの,多くはブラックボックスな分類に依存しており,研究者の探索が受動的になる恐れがある。そこで本研究では,説明可能な概念軸と発想の近さを同時に扱えるプラットフォームWISSpectrumを構築し,コミュニティ固有の議論を活性化する。

\section{関連研究}
\subsection{LLMを用いたサーベイ支援}
LLMや文書埋め込みを活用した論文リーディング支援が多数提案されているが,英語論文に偏っており,セクション構造や評価手法の比較まで踏み込む例は限定的である。日本語のHCI論文を対象とした支援も不足している。

\subsection{HCIにおける探索・可視化}
概念マップやネットワーク可視化などの研究支援は存在するが,論文を構成するセクションごとの差異を提示し,評価手法を比較する仕組みは少ない。WISSコミュニティ向けのコーパス整備も課題である。

\section{WISSpectrumの設計}
\subsection{設計目標}
WISSpectrumは次の3点を設計指針とした。
\begin{enumerate}
 \item \textbf{構造化要約}: 背景・目的・手法・評価を明示的に抽出し,研究差分を即座に把握できるようにする。
 \item \textbf{多層的探索}: 埋め込みによる意味距離とACM CCSによる概念距離を組み合わせ,枠組みに縛られない探索を支援する。
 \item \textbf{再現可能性}: PDF収集から可視化までの前処理を自動化し,コミュニティで再利用できるワークフローを整備する。
\end{enumerate}

\subsection{システム構成}
システムは前処理パイプラインと探索UIで構成される。前処理は以下の手順で実行する。
\begin{enumerate}
 \item \textbf{PDF収集}: WISS公式サイトのCSVを基に年度別に論文PDFを取得する。
 \item \textbf{テキスト抽出}: GROBIDによるTEI出力を優先し,失敗時はPyPDFと正規表現で疑似セクションを推定する。
 \item \textbf{セクション別要約}: OpenAI Responses APIを利用し,日本語と英語の二言語で背景・目的・手法・評価を要約する。
 \item \textbf{メタデータ整合性}: DOIや発行年を複数ソースで突合し,信頼度を \texttt{metadata\_meta} として保存する。
 \item \textbf{埋め込み計算}: Gemini Embedding APIを基盤とし,同一テキストをまとめて \texttt{batch\_embed\_content} に投入する。利用できない環境では逐次呼び出しにフォールバックする。
 \item \textbf{ACM CCS分類}: 埋め込み類似度から候補概念を抽出し,LLMで最終的な概念を選定する。
\end{enumerate}
探索UIではセクション別の埋め込み距離とACM CCS階層を可視化し,研究者が重み付けやフィルタを調整できるようにする。

\section{実装}
\subsection{前処理パイプライン}
前処理はPython製のコマンドラインツール群により実装した。 \texttt{summarize\_pdf.py} が個別PDFの抽出・要約・埋め込み・分類を一括で処理し, \texttt{orchestrator.py} がディレクトリ単位のバッチ処理を担う。出力はJSONで保存され, \texttt{embedding\_meta} や \texttt{metadata\_meta} を含む。

\subsection{埋め込みと類似度計算}
Gemini APIを利用する際はテキストを集約して重複計算を避け,バッチAPIが利用不可の場合は警告を出して逐次モードに切り替える。得られたベクトルはコサイン類似度で比較し,探索UIでの並び替えやクラスタリングに活用する。Vertex AIやSentence Transformersにも切り替え可能であり,ローカル検証を容易にした。

\subsection{インタラクション設計}
UIはD3.js等の可視化ライブラリを想定し,論文ノードを2次元マップ上にプロットする。ユーザは類似論文のグルーピングや評価手法の比較,英語要約を通じた国際的文脈の参照を行えるよう設計した。

\section{利用シナリオ}
\subsection{研究着想のブラッシュアップ}
研究者が自身のアイデアを入力し,類似論文の目的・手法・評価を参照することで,未踏の課題や差別化のポイントを発見する。

\subsection{査読準備}
プログラム委員や査読者が投稿論文に近い先行研究を迅速に把握し,背景・評価の差異を短時間で確認できる。

\subsection{教育支援}
大学院ゼミ等で学生が担当領域の研究系譜を学ぶ際,背景・目的・手法・評価のマッピングを通じて文脈理解を深める。

\section{考察と課題}
初期評価では,評価手法の比較や関連研究の俯瞰が容易になった一方, \texttt{call\_openai} のエラーに対するリトライ層が未整備であること,構造化ログやトークン使用量の記録が不足していることが指摘された。今後は堅牢性の向上とユーザスタディによる定量評価が必要である。また,埋め込みの永続キャッシュやユーザコメントを活用したフィードバックループも課題として残る。

\section{結論}
本稿では,WISSコミュニティを対象とした論文探索支援プラットフォームWISSpectrumを提案し,PDF収集から要約・埋め込み・分類・可視化までを統合したワークフローを示した。研究者が自身の着想を先行研究と結びつけながら議論の射程を見定めるための基盤として,今後はWebインタフェースの拡充とユーザ評価を進める予定である。

\end{document}
