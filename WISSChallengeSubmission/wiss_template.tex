\documentclass[twoside]{wiss}

\usepackage{ascmac}
\usepackage[dvipdfmx]{graphicx}
\usepackage{nidanfloat}
\usepackage{multicol}
\usepackage{color}
\usepackage{flushend}
\usepackage{url}

\journalhead{WISSpectrum: コミュニティの記憶を繋ぐ論文マッピングシステム}

\begin{document}

\title{WISSpectrum: コミュニティの記憶を繋ぐ論文マッピングシステム}
\etitle{}
\author{著者名\affil{所属}\thanks{著者情報は投稿時に更新すること.}}

\begin{abstract}
本研究はWISSコミュニティに蓄積された論文群の概念的距離を再編成し,研究者が過去の議論と現在の課題を往還できる論文マッピング支援システムWISSpectrumを提案する.既存のLLMベースのサーベイ支援が抱えるブラックボックス性に対し,ACM Computing Classification System(CCS)による構造的分類とVertex AI Gemini Embeddingによる意味的配置を併用することで説明可能性と探索性を両立する.ユーザは2次元マップ上で類似研究の近傍や時間的変遷を把握し,新規論文の位置づけを多角的に検証できる.本稿ではプロトタイプ実装を踏まえ,課題設定,設計方針,およびMethodセクションまでの論理構成を示す.
\end{abstract}

\maketitle

\section{はじめに}
Human-Computer Interaction(HCI)分野では,既存研究の再発見と新規テーマの構想を両立させるサーベイ作業が研究活動の鍵となる.しかしWISSのように長年継続してきたコミュニティでは,論文ごとの文脈や実装スタイル,議論の重心が時代とともに変化しており,単純なキーワード検索や引用関係だけでは背景を共有しづらい.さらに近年広がる大規模言語モデル(LLM)を用いた自動要約は,分析過程がブラックボックス化するため,創造性の源泉をどのように位置づけるかを研究者が追跡しにくいという問題が指摘されている.

WISSpectrumは,コミュニティが蓄積してきた記憶を概念的に再接続し,研究者が自身のアイデアの立ち位置を能動的に考察できる環境を提供することを目指す.本研究の狙いは以下の3点に整理できる.
\begin{itemize}
 \item 非自明な関連性の発見:時代・分野が異なる論文間の共通デザイン思想や応用領域を抽出し,新たな議論を誘発する.
 \item 研究概念の変遷の可視化:特定概念の系譜をデータに基づいて追跡し,WISSコミュニティ固有の流行や死角を明らかにする.
 \item 新規研究の位置づけ支援:投稿論文が既存研究とどのように接続し得るか,多層的な指標でフィードバックする.
\end{itemize}
本稿では,これらの狙いを達成するために必要なシステム要件と,実装済みの前処理基盤を中心としたメソッドを報告する.次章で課題背景を整理し,第3章で提案システムの全体像を述べる.第4章では,PDF解析・要約・概念タグ付け・埋め込み生成からなるパイプラインをMethodとして詳細に記述する.

\section{背景と課題}
研究サーベイの支援技術は,情報検索やトピックモデリング,ネットワーク可視化など多様に提案されてきた.しかしHCI領域では,論文が扱う現象,評価手法,ユーザインタフェースの設計思想が多岐にわたり,機械的に抽出したトピックのみでは議論の重心を適切に表現しにくい.WISSは「アイデアの斬新さ」や「未来ビジョン」を重視する査読文化を持つため,過去の議論と未来志向の思考とを往還しながら批評を行う枠組みが求められる.

一方でLLMを全面的に利用した自動分類に依存すると,モデル内部の判断根拠が不透明となり,研究者自身の探索行為が受動的になりかねない.本研究では「分類タスク」と「意味探索タスク」を切り分けることで課題に対処する.ACM CCSに基づく分類は有限集合へのマッピングであり,AIは制約付き分類器として機能する.一方,埋め込みベクトルは論文の思想を高次元空間上に表現するため,AIは意味的距離計として役割を持つ.この二層構造により,説明可能な軸と創発的な軸を両立させることが可能となる.

\section{提案システムWISSpectrum}

\subsection{コンセプトと要件}
WISSpectrumは,WISSで公開されている過去の予稿と,新たに投稿される論文ドラフトを同一空間上にマッピングするWebベースの探索環境である.システムは以下の要件を満たすよう設計する:\\
(1)\textbf{構造的信頼性}:ACM CCSに基づく概念ラベルを付与し,なぜ類似と判定したのかを説明可能にする.\\
(2)\textbf{意味的探索性}:Gemini Embedding(text-embedding-004)を用いて論文間の距離を高次元で計測し,既存枠組みに収まらない発想の近さを提示する.\\
(3)\textbf{コミュニティ連関}:過去の議論を再活性化するため,コメントやグルーピングを通じて研究者間の対話を誘発するインタラクションを備える.

\subsection{ユーザインタラクション設計}
ユーザは対象論文を入力すると,システムが2次元のマップ上に関連論文を可視化する.可視化はt-SNEやUMAPなどの次元削減を想定し,近傍ノードを選択することで以下の情報を参照できる:\\
(a)構造ランク:共有するCCS概念の重み付けに基づく近しさ.\\
(b)思想ランク:埋め込み空間におけるコサイン類似度.\\
(c)総合ランク:上記2指標を加重平均した総合スコア.\\
さらにユーザは類似論文を最大10件までグルーピングし,共通性(ビジョン,応用領域,技術進化など)を注釈できる.これにより,過去の議論を再文脈化し,新しい研究の方向性を促す「カードゲーム」的体験を提供する.

\section{Method}

\subsection{コーパス整備とACM CCS構造化}
対象コーパスはWISS予稿集(2003--2024年)を中心に収集し,PDFからテキストを抽出してメタデータを整形する.抽出後の文書は,人手による品質確認を前提として要約・タグ付け作業の素材とする.ACM CCSラベルは,LLMによる候補生成と,人間アノテータ(HCI,メディアコンテンツ,デザインの各分野に精通した3名)による照合を組み合わせる.アノテータは各ラベルの含有率(パーセンテージ)を評価し,構造ランク算出時に利用する重みを付与する.

\subsection{要約生成と概念タグ付けパイプライン}
前処理は`Pre-Processing/summary/summarize_pdf.py`に実装されているワークフローに依拠する.まず`pdf_extractors.py`でPDFを構造化し,章節ヘッダと本文を抽出する.続いて本文を意味的に一貫したチャンクに分割し,OpenAI Responses APIを用いたLLMサマライザで概要・目的・提案手法・評価の4視点に要約する.要約結果はヒューマンワーカーが事実性を確認し,修正後のテキストをJudge LLMに入力してACM CCSタグを推定する.この際,要約プロンプトとタグ付けプロンプトを分離し,分類タスクの制約性を高めている.

\subsection{意味埋め込みとマッピング生成}
要約テキストと重要節は`Pre-Processing/embeddings.py`で定義された`maybe_compute_embeddings_vertex_ai`を通じてVertex AI Gemini Embedding(text-embedding-004)によりベクトル化する.本番環境ではプロジェクトIDとリージョンを指定してVertex AIを初期化し,必要に応じて次元数を調整する.ローカル検証向けにはSentenceTransformersベースの`maybe_compute_embeddings_local`を用意しており,依存パッケージが存在しない場合は早期に警告を出す.得られた正規化済みベクトルはコサイン類似度を用いて比較し,近傍探索やクラスタリングに利用する.

\subsection{ランキング指標と更新フロー}
構造ランクは共有CCS概念の重み付き一致度として算出し,思想ランクは埋め込みベクトルのコサイン類似度に基づく.総合ランクは両者をパラメータ$\alpha$で線形結合したスコアとし,ユーザが探索目的に応じて重みを調整できるようにする.ランキング算出後は,想定利用者によるフィードバックループを設計し,推奨結果に対するコメントやグルーピングをメタデータとして蓄積する.この情報を再学習時に活用することで,コミュニティに即した意味空間を継続的に更新する.

\subsection{プロトタイプ実装の現状}
現時点ではPDF要約からベクトル生成までのバッチ処理が整備済みであり,`.env`によるAPIキー管理やチャンク化ロジックなど,運用に必要な設定がスクリプト内に組み込まれている.今後はWebインタフェースとインタラクティブなマップ描画,ユーザコメント機能,および評価実験の設計へと拡張する予定である.

\end{document}
